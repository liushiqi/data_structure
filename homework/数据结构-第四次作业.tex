\documentclass[11pt]{article}

\usepackage[headheight=14pt,a4paper]{geometry}
\geometry{left=2.0cm,right=2.0cm,top=2.5cm,bottom=2.5cm}

\usepackage[UTF8]{ctex}
\usepackage{tikz}
\usepackage{tikz-qtree}
\usepackage{pstricks}
\usepackage{amsfonts,graphicx,amssymb,bm,amsthm}
\usepackage{algorithm,algorithmicx}
\usepackage[noend]{algpseudocode}
\usepackage{fancyhdr}
\usepackage[fleqn]{amsmath}
\usepackage{listings}

\newenvironment{solution}{{\noindent\hskip 2em \bf 解 \quad}}

\renewenvironment{proof}{{\noindent\hskip 2em \bf 证明 \quad}}{\hfill$\qed$\par\vskip1em}
\newenvironment{proofs}[1]{{\noindent\hskip 2em \bf #1 \quad}}{\hfill$\qed$\par\vskip1em}

\makeatletter

\let\old@@children\@@children
\def\@@children{\futurelet\my@next\my@@children}
\def\my@@children{%
\ifx\my@next\missing\else
\expandafter\@gobble
\fi
\expandafter\old@@children}

\makeatother

\newcommand{\missing}{ \edge[draw=none]; \node[draw=none]{}; }
\newcommand{\tabincell}[2]{\begin{tabular}{@{}#1@{}}#2\end{tabular}}

\begin{document}

\pagestyle{fancy}
\rhead{\kaishu 数据结构}

\begin{center}
  {\LARGE{\bf{数据结构作业}}}\\
  {\large{\bf{第三次}}}\\
\end{center}

\begin{flushright}
  \begin{kaishu}
    姓名:刘士祺 \\
    学号:2017K8009929046 \\
  \end{kaishu}
\end{flushright}

\begin{enumerate}
  \item[5.] 习题8-1
    \begin{enumerate}
      \item [1)]
        \begin{solution} \\
          \input{8-1-1.tex}
        \end{solution}
      \item [2)]
        \begin{solution} \\
          \input{8-1-1.tex}
        \end{solution}
    \end{enumerate}
    
  \item[6.] 习题8-7

  \item[7.] 习题9-1
    
  \item[8.] 习题9-14

  \item[9.] 习题9-19

  \item[10.] 习题9-24

  \item[11.] 习题10-1

  \item[12.] 习题10-3

  \item[13.] 习题10-15

  \item[14.] 习题10-21

  \item[15.] 习题11-1

  \item[16.] 习题11-2

  \item[17.] 习题11-5

  \item[18.] 习题11-11
\end{enumerate}
\end{document}
